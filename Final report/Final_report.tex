\documentclass[conference]{IEEEtran}
\IEEEoverridecommandlockouts
% The preceding line is only needed to identify funding in the first footnote. If that is unneeded, please comment it out.
\usepackage{cite}
\usepackage{amsmath,amssymb,amsfonts}
\usepackage{algorithmic}
\usepackage{graphicx}
\graphicspath{{img/}}

\usepackage{textcomp}
\usepackage{xcolor}
\def\BibTeX{{\rm B\kern-.05em{\sc i\kern-.025em b}\kern-.08em
    T\kern-.1667em\lower.7ex\hbox{E}\kern-.125emX}}
\begin{document}

\title{Computer Project I -- Final Report} 

\author{\IEEEauthorblockN{1\textsuperscript{st} Asger Song}
\IEEEauthorblockA{\textit{Aarhus University} \\
\textit{Deparment of electrical and computer engineering}\\
Study number: \\
AU-id: }
\and
\IEEEauthorblockN{2\textsuperscript{nd} Andreas Kaag Thomsen}
\IEEEauthorblockA{\textit{Aarhus University} \\
\textit{Department of electrical and computer engineering}\\
Study number: 202105844 \\
AU-id: au691667}
}

\maketitle

\begin{abstract}
    This is our abstract.
\end{abstract}

\section{Introduction}
    Overall, the project can be divided into four parts with certain aim and objectives: 
    \begin{enumerate}
        \item Understanding the basic concepts of robot programming by working with an Arduino - that is, the \emph{think-see-act cycle}. 
        Reading and analyzing data from Range sensor and RGB sensor.
        \item ROS Programming on Raspberry PI. Learning the structure that ROS provides and going through the beginner's tutorial. 
        \item Programming the robot to avoid obstacles. Then optimize the robot's performance both with respect to linear speed and collision avoidance.
        \item Finalizing the code and testing the robot on an obstacle course. 
    \end{enumerate}

\section{Specifications}
We have used the following equipment throughout the course. 
\begin{itemize}
    \item Arduino PRO MICRO - 5V/16MHZ 
    \item Ultrasonic Range Finder (LV-MAXSONAR-EZ0)
    \item RGB Light Sensor ISL29125
    \item LED's, cables etc.
    \item Turtlebot3 Burger Robot equipped with i.a. a Raspberry Pi 3 and a 360\textdegree LiDAR sensor.  
\end{itemize}
For coding we have used the language C for programming the Arduino on the Arduino software. 
For programming the Turtlebot we have been using Python and the command prompt with the built in nano-editor. 

\section{Design and implementation}
    Design and implementation of the system.

    \subsection{Part 1}


    \subsection{Part 2}


    \subsection{Part 3}


    \subsection{Part 4}

\section{Experiment setup and results}

\section{Discussion}
In this section of the report we will discuss some of the difficulties we encountered during the project.
Again, we have divided it into the four main parts of the project described earlier. 

    \subsection{Part 1}


    \subsection{Part 2}


    \subsection{Part 3}


    \subsection{Part 4}

\section{Personal contributions}

\section{References}



\end{document}